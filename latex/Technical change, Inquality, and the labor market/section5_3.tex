\documentclass[dvipdfmx]{beamer}
\usetheme{Boadilla}
\usecolortheme{whale}
\usepackage{umoline}
\usepackage{graphicx}% 各種画像の張り込み
\usepackage{amsmath,amssymb}%標準数式表現を拡大する
\usepackage{color}
\usepackage{multicol}
\usepackage{otf}
\setbeamertemplate{navigation symbols}{}
\AtBeginSection[]{
    \frame{\tableofcontents[currentsection, hideallsubsections]} %目次スライド
}
\setbeamertemplate{footline}[frame number]
\usepackage{chngpage}
\usepackage{caption}
\newcommand{\bhline}[1]{\noalign{\hrule height #1}}
\newcommand{\bvline}[1]{\vrule width #1}

%----------- ここからスライド内容 ---------------%
\title[]{\small Technical Change, Inequality, and the Labor Market}
\subtitle{§5 Acceleration in Skill Bias\\ -A Puzzle: The Decline in Wages
of Low-Skill Workers-}

\author[]{Ichiro Kozakai}
\date{\today}


\begin{document}
\begin{frame}
  \titlepage 
\end{frame}

\begin{frame}
  \begin{itemize}
    \item This sub-section deals with why the wage for the unskilled
    is relatively decreasing, leading to wage-premium expansion.  \vspace{0.1in}
    \begin{itemize}
      \setlength{\itemsep}{0.15in}
      \item According to the theory mentioned in 3rd section, the increasing supply of
      educated workers raises the productivity of high-skill workers, $A_{h}$, leading to the wages up as a whole.
      \item The former papers discuss the faster technology progress lower the skill premium, since the technology seems to work as
      making the productivity gap, $ \frac{A_{h}}{A_{l}}$, smaller.
      \item This technology revolution in these past decades, however, works in the verse way, boosting the relative productivity, $\frac{A_{h}}{A_{l}}$.
      \begin{itemize}
        \item[$\Longrightarrow$] $\ln \omega =\frac{\sigma -1}{\sigma}\ln (\frac{A_{h}}{A_{l}})-\frac{1}{\sigma}\ln (\frac{H}{L})$, which results in higher, \uncover<2->{\color{red}"erosion effect"}.
      \end{itemize}
    \end{itemize}
  \end{itemize}
\end{frame}

\begin{frame}{Repost}
  \begin{equation}
    \begin{split}
      A_{l}=\phi_{l}(g)a\\
      \phi_{l}^{\prime}<0\\
      A_{h}=\phi_{h}a        \notag
    \end{split}
  \end{equation}
\end{frame}



\begin{frame}
  \begin{itemize}
    \setlength{\itemsep}{0.2in}
    \item The wage for unskilled workers is re-written as 
    \begin{equation}
      \begin{split}
        w_{L}=\phi_{l}(g)a[1+\phi_{h}^{\rho}\left(\frac{H}{L}\right)^{\rho}]^{\frac{1-\rho}{\rho}}     \notag
      \end{split}
    \end{equation}  \vspace{0.15in}
    \begin{itemize}
      \setlength{\itemsep}{0.1in}
      \item[$\Longrightarrow$] The growth of wage for the lower skilled is; $\frac{\dot{w_{L}}}{w_{L}}=g(1+\epsilon_{\phi})$\\ \hspace{0.5in} \footnotesize $\epsilon$: the elasticity of of the function, $\phi_{l}$, and $\phi_{l}^{\prime}<0$.
      \item[$\Longrightarrow$] If the value of $\epsilon$ is less than -1 ($\epsilon<-1$), the  wage for low-educated worker is absolutely decreasing. 
    \end{itemize}
    \item Firm's positive actions to technology cause the capital-labor ratio for the low-educated,
    with the wage for the less-skilled falling.
    \begin{itemize}
      \setlength{\itemsep}{0.1in}
      \item[$\Longrightarrow$] Because the equilibrium rate of return to capital increases.\hspace{0.2in} (by Cassel)
      \item[$\Longrightarrow$] Because the firms devotes more resources to opening specialized jobs for the high-skilled.  \hspace{0.2in} (in this paper) 
    \end{itemize}
  \end{itemize}
\end{frame}




\begin{frame}
  \begin{itemize}
    \item<1-> The production function is defined as following;\\
  \end{itemize}
  \vspace{0.15in}
  \begin{equation}
    \begin{split}
      Y_{h}=K_{h}^{1-\alpha}(A_{h}H)^{\alpha}\\
      Y_{l}=K_{l}^{1-\alpha}(A_{l}L)^{\alpha}\\
      \bar{K}=K_{h}+K_{l}            \notag
    \end{split}
  \end{equation}
  \hspace*{0.25in} $Y_{l}$ and $Y_{h}$ are perfect substitutes.
  \vspace{0.15in}
  \begin{itemize}
    \item<2-> At equilibrium point, the $MP_{l}$ in each type of labor is balanced;
    \begin{equation}
      \begin{split}
        \frac{K_{l}}{A_{l}L}=\frac{K_{h}}{A_{h}H}=\frac{K-K_{l}}{A_{h}H}   \notag
      \end{split}
    \end{equation}
    \item<2-> This is rewritten as following;
    \begin{equation}
      \begin{split}
        \frac{\bar{K}}{K_{l}}-1={\frac{A_{h}}{A_{l}}}{\frac{H}{L}}    \notag
      \end{split}
    \end{equation}
  \end{itemize}
\end{frame}


\begin{frame}
  \begin{itemize} 
    \setlength{\itemsep}{0.15in}
    \item The wage for the low-skilled is written under the condition of perfectly competitive labor market,
    \begin{equation}
      \begin{split}
        w_{L}=MP_{L}=(1-\alpha)A_{l}^{\alpha}K_{l}^{1-\alpha}L^{\alpha-1}     \notag
      \end{split}
    \end{equation} 
    and this is falling.
    \end{itemize}
\end{frame}

\begin{frame}
  \begin{itemize}
    \setlength{\itemsep}{0.15in}
    \item Potential problems in this analysis are
    \vspace{0.05in}
    \begin{itemize}
      \setlength{\itemsep}{0.1in}
      \item This theory is based on the assumption of the increase of price of capital
      \item Financial Policy raises intentionally the interests rate.
      \begin{itemize}
        \item[$\Rightarrow$] Future studies will clarify that the causal relation between higher interests rate and lower wage for the uneducated.  
      \end{itemize}
    \end{itemize}
    \item The next section explains that the effect of technical
    change on the organization of the labor market both amplifies the effect of
    technology on wage inequality, and provides a possible explanation for this decline.
  \end{itemize}
\end{frame}
\end{document}
