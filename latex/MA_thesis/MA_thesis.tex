\documentclass{jsarticle}
\usepackage{indentfirst}
\usepackage[dvipdfmx]{graphicx}
\usepackage{url}
\usepackage[dvipdfmx]{hyperref}
\usepackage{pxjahyper}
\title{日本市場におけるM\&A}
\author{細岡健司\thanks{慶應義塾大学経済学部千賀達朗研究会1期生}\and 細野裕介
\footnotemark[1] \and 小酒井一朗\footnotemark[1]}
\date{\today}
\begin{document}
\maketitle
\tableofcontents
\section{はじめに}
1929年の世界恐慌とそれからの復興期より主要資本主義工業国では経済学者 J.Keynes\footnote[1]{John Maynard Keynes.}の提唱による大きな政府が図られてきたが、1980年代初頭のM.Thatcher英大統領、R.Reagan米大統領、中曽根首相により新自由主義に舵を切った。日本では三公社民営化\footnote[2]{行財政改革の一環として、1985年に日本専売公社、日本電信電話公社が、1987年に日本国有鉄道が民間経営移管された。}どが図られ、小さな政府と自由な市場が継起した。また、1986年10月に英国にて金融ビッグバンが起こると、市場経済と金融市場の両市場が開かれることで、経済のグローバル化が進展する。それにプラザ合意・ルーブル合意による円高傾向も手伝い、バブル期を中心に日系企業の対外投資・クロスボーダーM\&Aが隆盛した。しかし、小さな政府化が市場の企業吸収・合併を加速させたのはこの時が初めてではなかった。19世紀末ごろの日本において、殖産興業事業にて民間事業化が育つと、当時の政府は官営事業の払い下げを行うことによりM\&Aが活発に始まり、後の財閥形成の流れが築かれた。ここでは2例のみを取り上げたが、共通していることは、国内マクロ経済が好調な際に企業の吸収・合併が隆盛しているということだ。\par
このように、M\&Aは日本においてもそれほど目新しい事象ではなく、資本主義経済史と同等の歴史を有している。そのため、M\&Aに関する研究は既に取行われているが、M\&Aが行われる背景や経済的役割に関する理論形成が主であり、M\&Aによる企業活動への影響や、マクロ経済の構造調整に関する実証研究の数は乏しく思われる。そこで本稿はDavid(2020)を参考とし、Bloombergが公開する20世紀末から2021年上半期まで(1993年3月19日~2021年10月4日)の日本市場における取引額1Mドル以上のM\&A案件\footnote[3]{本稿では、IN/-INとOUT-IN型のみを対象とした。}について、買収・被買収企業の利益性、財務レバレッジ、現金比率、売上高販管費率に関する財務諸表データを対象とした上で、両者の企業の関係性とM\&A取引の特性を明らかにする。更に、M\&Aの性質をIN-IN M\&A(日本企業による日本企業の買収)とOUT-IN M\&A(アメリカ企業による日本企業の買収)に分別した上でそれぞれ独立的に分析を行った。\par
第2節では米国のM\&Aに関する先行研究であるDavid(2020)の結果との比較から、日本市場のM\&Aにおける共通点や独自性に関して明らかにし、第3節では買収・被買収企業の財務項目に焦点を置き、日本市場のM\&Aに出現する買収・被買収企業の関係性を整理する。\par
最後に、本稿の執筆にお力添えして頂いた千賀准教授にはこの場を借りて感謝申し上げる。


\section{米国との比較}
この節では、David(2020)をレプリケートし、日米市場におけるM\&Aの特性を比較検討していく。表1\footnote[4]{David(2020)より引用}によると米国における1977~2009年のM\&Aの取引額は平均が\$267.4M、中央値が\$31Mと右に歪んだ分布となっている。一方、表2の日本市場のM\&Aでは、それぞれ\$129.9Mと\$15.1Mとなっており、同様の傾向が見受けられる。次に、取引における発表プレミアムは米国では47%と38%となっており、平均値と中央値が近しい。一方の日本は19.5%と13.1%となっており、ここにも同様の傾向が見られる。このプレミアムとは、会計上ののれんに近しく、技術力や人的資源、顧客ネットワーク等の被承継企業の超過収益力を表し、無形資産として計上される。日米における発表プレミアムを比較した際に、日本市場のプレミアム値が相対的に低くなっている。つまり、日本における企業吸収・合併は総取引額に占める純資産額率が高く、有形資産の比率が高い。そのため、日本市場と米国市場における企業吸収・合併における買収側企業の意図には相違があると考えられ、この点に関しては第4節にて扱う。 \par
\begin{table}[http]
\begin{center}
\caption{Transaction Values and Premium in US}
\begin{tabular}{lcrr}
\hline
&& Trans. Val. (US\$M) & Premium (\%) \\  \hline
Mean && 267.4 & 46.8 \\
Median && 31.0 & 38.2 \\
SD && 1,911.5 & 34.8 \\
Max && 186,824.1 & 200.0 \\
Min && 0.9 & 0.0 \\ \hline
N && 57,858 & 5,868 \\ \hline
\end{tabular}

\end{center}
\end{table}

\begin{table}[http]
\begin{center}
\caption{日本のIN-IN型 M\&A取引}
\begin{tabular}{lcrr}
\hline
&& Trans. Val. (US\$M) & Premium (\%) \\  \hline
Mean && 129.9 & 19.5 \\
Median && 15.1 & 13.1 \\
SD && 958.0 & 33.1 \\
Max && 44,815.1 & 376.2 \\
Min && 1 & -84.6 \\ \hline
N && 10,929 & 1,617 \\ \hline
\end{tabular}
\end{center}
\end{table}
また、David(2020)によると、1977~2009年のアメリカ市場では、1度M\&Aを行った企業はその後も
\begin{table}[http]
\begin{center}
\caption{Distribution of Transactions by Number of Acquirer Purchases}
\begin{tabular}{lccc}
\hline
Number of Purchases&Firms&Transactions&Share of Transactions\\  \hline
1&18,870&18,870&32.6\\
2-3&5,777&13,301&23.0\\
4-5&1,669&7,345&12.7\\
6-7&733&4,695&8.1\\
8-10&490&4,311&7.5\\
11-15&340&4,251&7.3\\
16-20&119&2,100&3.6\\
More than 20&107&2,985&5.2\\ \hline
Total&28,105&57,858&100.0\\  \hline
\end{tabular}
\end{center}
\end{table}
\begin{table}[htbp]
\begin{center}
\caption{日本市場の吸収・合併回数による買収企業の分布}
\begin{tabular}{lccc}
\hline
Number of Purchases&Firms&Transactions&Share of Transactions\\  \hline
1&2,422&2,422&20.0\\
2-3&1070&2,457&20.2\\
4-5&283&1,247&10.3\\
6-7&138&887&7.3\\
8-10&86&760&6.3\\
11-15&63&798&6.6\\
16-20&31&569&4.7\\
21-30&30&749&6.2\\
31-40&11&381&3.1\\
41-50&5&227&1.9\\
More than 50&11&808&6.7\\
Multiple buyer&&194&1.6\\
Unnamed&&638&5.3\\  \hline
Total&4,150&12,137&100.2\\  \hline
\end{tabular}
\end{center}
\end{table}\\
M\&Aを繰り返す事が分かったが、日本のIN-IN市場でも同様の傾向が見られる。表3\footnotemark[4]と表4は米国と日本において1買収企業が何度M\&Aを行っているかを表したものである。アメリカ市場ではそれでも初経験の企業が多く、11回以上吸収・合併を行う企業は20\%にも満たない。それに対し、日本市場では29\%を占め、これはアメリカ市場の約2倍となっている。往々にして、アメリカでは独占禁止法による拘束力が他の地域と比較すると相対的に緩く、市場占有度の不均衡に寛容的と言われる。しかし、実際には日本の方が市場の不完全競争市場に傾きかけていると言える。宮島(2006)にて示唆されている通り、M\&Aの目的としては数種類存在し、主たる目的は買収企業の新規事業への進出・マイナー事業の拡大と、同業界のライバル企業買収を通したノウハウ獲得や企業価値の伸張の2種である。上述の通り日本市場におけるM\&Aにおいては無形資産の価値計上が低く、有形資産比率が高いことから、日本では企業のコングロマリットが主たる動機であると考えられる。それ故、M\&Aが買収側企業の主幹事業の業界市場占有率や資本・労働力の再分配に与える影響度は比較的小さいと言える。日系企業がM\&Aを通してコングロマリットを拡大した背景としては、バブル崩壊後の経済成長停滞と金融業の貸し渋りがあげられる。1990年頃に土地バブルが崩壊すると、金融機関が多額の不良債権を抱え、更には国際法としてのBIS法\footnote[5]{バーゼル規制やバーゼル合意ともいわれ、銀行の国際活動には自己資本比率が8\%以上であることを条件とした。}が制定されたことを受け、金融機関は貸し渋りを行った。それにより、中小企業をはじめとした日系企業への融資額が低下し、内部留保化が促進された。一方で、バブル崩壊後の日本経済の停滞は内需の停滞を引き起こし、企業の多角的事業化をも促した。また、コングロマリット化は企業株式の安定化をもたらすため、企業価値の停滞につながると言われるが、これによってエクイティを用いた資本獲得をも困難化させた。故に、新規事業への新規投資が困難となり、既存企業の吸収・買収へと舵をきったのであろうと考えられる。\\






\section{日本市場におる買収・被買収企業の関係性}
この節では、日本市場におけるIN-IN型M\&Aに特化し、買収側企業と被買収側企業の特性に関して考察した後にOUT-IN型に関しても簡単に触れる。図1\footnotemark[4]はアメリカの1977年から2009年のM\&Aにおける買収側企業と被買収側企業の利益に関する二次元分布である。一方、図2は日本のIN-IN型M\&Aにおける2社のEBIDTA\footnote[6]{国によって異なる金利水準、税率、減価償却法の影響を小さくし、企業収益力の国際比較を容易化させる指標である。具体的には税引前利益に支払利息、減価償却費を加えることにより算出される。}に関する2次元分布となっている。図2-Aは2社ともにEBIDTAが正の値である事例のみを抽出し、その値を対数化させた分布を示しており、図2-Bは(A)をさらに、買収側・被買収側に関して標準化した値の分布となっている。図2-Cは同事例に関するそれぞれの値をBoxCox\footnote[7]{吉田・松井(2019)参照}変換した値をプロットしたものであり、図2-Dは利益が負である事例に関しても含め、Yeo-Johnson\footnotemark[7]変換した値をプロットしたものである。まず、図1と図2-A・Bを比較すると、両国市場における殆どの事例で、買収側企業の利益やEBIDTAが被買収側企業のそれを上回っていて、いわゆるhigh-buys-low(q理論)型合併\footnote[8]{生産性の高い企業が低い企業を吸収・合併することにより、生産・経営資源を低生産性企業から高生産性企業へと移転すること。}である。しかし、米国市場と比較すると、日本でのM\&Aでは収益力の低い被買収企業に対する買収企業の収益力がかなり高いことが分かり、一層のhigh-buys-lowの傾向も認識できる。一方で、米国市場でのM\&Aにおける収益力の低い(被買収)企業の買収事例に関してはlike-buys-likeの傾向が出現する。\par
また図2-Cと図2-Dを比較すると、赤字企業の事例を含めることでより一層収益力の低い(被買収)企業の買収ではhigh-buys-lowの傾向が強く、収益力の高い(被買収)企業の買収では、like-buys-likeの傾向が強いと言える。ここで、説明変数と被説明変数に関し、それぞれ標準化を行った分布における線形回帰の回帰係数は説明変数の変化に対する被説明変数の変化の感度を示す。図1と図2-Aを見比べると、Best-Fitと45 Degree Lineの交点の座標が殆ど同じであることから、回帰係数の差異は低収益力(被買収)企業の買収に関しての差異として還元される。日本市場においては、低収益力(被買収)企業の買収事例であるほど、45 Degree Lineから上方に外れている事が見て取れるため、企業間の収益力の差異が如実に表れていると言える。また日本市場での買収側企業の収益力は米国市場に比べて相対的に高い必要があり、本稿冒頭に触れた通り、これによってM\&A事象数と日本経済の関係は正であることが示唆される。同様に図2-Cと図2-Dを比較することで、赤字(被買収)企業の買収においては上述の傾向が一層高いことが確認できる。\par
\begin{figure}[ht]
\centering
\includegraphics[keepaspectratio,scale=0.45]{DAVID_figure1.png}
\caption{Joint Distribution of Acuquirers and Targets in US}
\end{figure}
\begin{figure}[htbp]
\centering
\includegraphics[keepaspectratio,scale=0.55]{output_EBIDTA.png}
\caption{日本のM\&Aにおける買収側と被買収側のEBDTAの二次元分布}
\end{figure}\clearpage
図3は日本での事象に関する買収・被買収企業の財務レバレッジをそれぞれ変数変換しプロットしたものである。ここでは総負債(総資本)と自己資本より自己資本比率の逆数として財務レバレッジを算出した。このとき、買収側企業の財務レバレッジが被買収側企業のそれを上回る事例の割合は、両企業共に財務レバレッジが正である事例の中では約56\%、全事例の中では約57\%であった。このことからマクロ的な視点では、2社間の財務レバレッジ比率の大小による趨勢はない。しかし、図3-Bを詳細に見てみると、財務レバレッジの低い(被買収)企業の買収事例における買収企業のそれは被買収企業のそれより総じて高く、一方で財務レバレッジの高い(被買収)企業の買収事例における買収企業のそれは被買収企業のそれより総じて低くなっている。ただし、2社間の財務レバレッジは正の相関となっている。財務レバレッジは企業の成長度を反映する一方で、企業経営の危険度をも反映したものである。また、企業の成熟度は企業の成長度に反比例する傾向が強い。これらの事実に加えて上述の考察を鑑みると、企業経営リスクが高いが成長度の高い企業が企業吸収・合併を行う際には、成熟しているが、自社より成長度が鈍っている企業(事業)の買収を好む傾向にあると言える。その一方で、経営が安定的で成長度が低いが成熟度の高い企業が企業吸収・合併を行う際には、他産業のマイナーである(成熟度が低い)が自社に比べ相対的に成長度の高い企業をターゲットとして好むと考えられる。更に、財務レバレッジを企業の信頼度と見なし、財務レバレッジの高さが多角的事業を行っている企業または成長度の高さであると捉えれば、上述の内容と合わせて次のようにも考察できる。成長度が高く、新規事業展開に意欲的な専門性の高い企業が、コングロマリットを目的として、成熟度が高く成長度の低い企業の一部事業や部門を吸収・合併していると考えられる。一方、成熟度が高く成長度の低い多角的経営企業が新規事業開拓目的のために、小規模で成長度が高く、専門性の高い企業を買収していると考察できる。そして、既存事業の成長度が鈍り始めると、さらなるコングロマリットを形成する傾向が強いと考えられる。実際に財閥系企業を中心とした総合商社の事業拡大モデルと似つかわしい。図3-Cと図3-Dを見比べると、負債超過企業の買収事例をも含め同様の傾向を示している。また、負債超過企業を買収している企業は経営が安定している多角的企業、または成長度と専門性の比較的高い企業であると考えられるが、恐らく前者の傾向が強いと考えられる。\par
\begin{figure}[ht]
\centering
\includegraphics[keepaspectratio,scale=0.5]{output_financial_leverage.png}
\caption{買収・被買収企業間の財務レバレッジ}
\end{figure}
図4は日本におけるIN-IN型M\&A事例における両企業の現金比率を対数化した値の組み合わせを示したものである。全事例の内、買収側企業の比率が上回った事例は約55\%であるが、図の左側でその傾向が強く、右側では弱くなっている。ここで現金比率を流動負債(短期負債)に占める現預金の比率と定義に従うと、現金比率の高さは現金預金の高さまたは流動負債の低さ若しくはその両方を意味する。仮に流動負債を借入金と考えたとき、後者の解釈に従うと成熟度の高い多角的企業ほど内部留保率が高いと考えられ、そういった企業は図の右側より、現金比率の高い企業とのM\&Aを好む一方で、図の左側より、現金比率の低い企業は一層現金比率の低い企業とのM\&Aを好む。つまり、事業拡大を図る大企業は成長度の鈍い大企業とのM\&Aを好むか、またはIT企業などの借入金額は小さいが専門性の高い成長企業を好むと考えられる。一方で、借入依存的な成長企業はM\&A相手として、同じく借入投資依存型の企業とのM\&Aによる主幹事業の拡大またはコングロマリット化を意図すると考察できる。\par
\begin{figure}[ht]
\centering
\includegraphics[keepaspectratio,scale=0.75]{output_cashratio.png}
\caption{買収・被買収企業の現金比率}
\end{figure}
図5は日本市場のIN-IN型M\&A事例に関する、買収企業と被買収企業の売上高販管費率\footnote[9]{$(1-営業利益率)×粗利益率×100$にて売上高販管費率を算出。}の対数値をプロットしたものである。売上高販管費率は営業活動の効率性を判断する指標であり、これが低いほど企業の採算性が高いことを意味する。データが算出できた事例の内、買収側が被買収側より採算性が高かった事例は約36\%であった。また、被買収企業の売上高販管費比率が小さい事例(被買収企業の採算性が高い事例)では買収企業の方が相対的に採算性は低く(図の45度線より上側の事例)、採算性の低い企業の吸収・合併事例(図の右側)では買収企業の採算性の方が高い(図の45度線より下側の事例)傾向が顕著である。これまでの考察も含めると、主に日本のM\&Aはコングロマリット化またはそれによる新規事業の拡大を意図し、経営の採算性が低い時にはより経営効率の高い企業を買収する。経営の採算性が高くなってくると、経営効率の低い企業を買収する。すなわち、市場においてマイノリティである勢力時には、当該事業での勢力が少々格上の小企業を吸収・合併し、市場での勢力が大きい時にはライバル企業や中企業を吸収・合併している事が分かる。\par \clearpage
\begin{figure}[h]
\begin{tabular}{cc}
\begin{minipage}[t]{0.45\hsize}
\centering
\includegraphics[keepaspectratio,scale=0.6]{output_SGAratio.png}
\caption{買収・被買収企業の売上高販管費率}
\end{minipage}&
\begin{minipage}[t]{0.45\hsize}
\centering
\includegraphics[keepaspectratio,scale=0.65]{output_outin_ebidta.png}
\caption{買収・被買収企業のEBIDTA}
\end{minipage}
\end{tabular}
\end{figure}
図6は日本市場のOUT-IN型M\&A事例においてEBIDTAが2社ともに正であるものに関し、両企業ともに対数値を取りプロットしたものである。132件中126件が買収企業の方が高収益力であった。また、高収益力企業ほど高収益力企業を買収している事が分かり、その傾向の強さもIN-IN型M\&Aに見られるものと同じである。これは図6の相関係数と図2-Aの相関係数の類似性から明らかである。しかし、OUT-IN型M\&Aは事例数が過小であるため、更なる追加検証考察が必要である。
\section{おわりに}
ここまで日本市場におけるIN-INとOUT-INのM\&Aに関して検証考察してきたが、それらからIN-INについて分かったことを整理すると、次の通りである。\\
\begin{itemize}
  \item 日本市場は米国市場に比べ相対的にM\&Aのプレミアム(金銭化の難しい無形資産)の評価が低く、このことからM\&Aの主目的はコングロマリットである。
  \item 1度事業拡大を図った企業はその後も繰り返し拡大する。
  \item 高収益力の企業が低収益力の企業をM\&Aしている。収益力の差は低収益企業のM\&Aほど強いが、2企業間の収益力には正の相関関係がある。
  \item M\&Aによって事業拡大を行う際には、成熟度が高く成長度が低い企業は自社より成長度の高い企業を相手にするが、経営リスクを伴っているが、成長度の高い企業は自社より成長度の乏しい企業を相手に選定する。
  \item 採算性の低い企業は自社より高い企業をM\&A相手にし、採算性の高い企業は自社より採算性の低い企業をM\&A相手にする。また、採算性の高い企業ほど採算性の高い企業をターゲットとする。
\end{itemize}
これらの事実から、成長度の高い企業がまずは成長度が相対的に乏しいが採算性の高い企業をM\&Aすることで、収益力を引き上げ、市場占有度を上げる。そして、行き詰まりを感じると、他分野に進出し、同じ事を繰り返す。これらを繰り返すことで企業経営の多角化を成し遂げていると考えられる。これを実証するためには、M\&Aの市場占有度の変化を考察する必要があり、それは今後の課題である。
\begin{thebibliography}{9}
  \bibitem{Miyajima}宮島英昭、「急増するM\&Aをいかに理解するか:その歴史的展開と経済的役割」、 \\\url{https://www.rieti.go.jp/jp/publications/dp/06j044.pdf}
  \bibitem{Yoshida, Matsui}吉田拓倫・松井藤五郎、「血液検査データに対する包括的な分類分析のためのデータ変換法」、"The 33rd Annual Conference of the Japanese Society for Artificial Intelligence, 2019"
  \bibitem{David}Joel M.David, "The Aggregate Implications of Mergers and Acquisitions",\\\url{https://doi.org/10.1093/restud/rdaa077}
\end{thebibliography}
\end{document}
