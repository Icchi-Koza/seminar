\documentclass{article}
\usepackage{indentfirst}
\title{Skill-Biased Technical Change}
\author{Ichiro Kozakai}
\date{\today}

\begin{document}
\maketitle
\section{Summary}
 This paper posts  a clue to whether technological capital and labor force are complemental or substituted. The answer is that technologies, especially information technologies, raise the productivity with skilled labor in a harmonious way and unskilled in a discordant way, which is called Skill-Biased and is a reverse to historical trend.\\
  The capital skill becomes cheaper and cheaper, which promotes equipment stock introduced into production process more, leading to more relative demand for skilled labor force that is good compatibility with tech capital. In second phase, the more skilled worker can adapt flexibly, leading employment of the highly educated to reducing training cost for firms. At last phase, information technologies has brought the "shift", very compatible with the high-skilled, into organizational managements. The difference in productivity brought by this trend leads to inequality  through the rise of relative wage.
\end{document}
